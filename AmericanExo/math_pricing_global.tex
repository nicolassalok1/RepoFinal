\documentclass[11pt,a4paper]{article}
\usepackage[utf8]{inputenc}
\usepackage[T1]{fontenc}
\usepackage[french]{babel}
\usepackage{amsmath,amssymb,amsfonts}
\usepackage{bm}
\usepackage{geometry}
\usepackage{hyperref}
\geometry{margin=2.5cm}

\title{Notes de cours sur le pricing d'options\\[0.3em]
PDE (Crank--Nicolson), Monte Carlo, Longstaff--Schwartz, arbres CRR, barrières et lookbacks}
\author{}
\date{}

\begin{document}

\maketitle
\tableofcontents
\newpage

%======================================================================
\section*{Avant-propos}
\addcontentsline{toc}{section}{Avant-propos}

Ces notes résument et expliquent, de manière pédagogique, les méthodes de
pricing implémentées dans le projet :
\begin{itemize}
  \item schéma de Crank--Nicolson pour la PDE de Black--Scholes
        (options européennes, américaines, bermudéennes, barrières) ;
  \item méthodes de Monte Carlo (classique et Longstaff--Schwartz) ;
  \item modèle binomial de Cox--Ross--Rubinstein (CRR) ;
  \item pricers analytiques et numériques pour options barrières et lookback.
\end{itemize}

L'objectif est double :
\begin{enumerate}
  \item donner un cadre mathématique \emph{clair} (équations, hypothèses) ;
  \item faire le lien avec la \emph{mise en œuvre numérique} telle
        qu'elle apparaît dans le code et dans l'application Streamlit
        \texttt{pricing\_app.py}.
\end{enumerate}

Le lecteur peut parcourir ces notes comme un mini cours d'introduction
au calcul stochastique appliqué à la finance de marché.

\newpage

%======================================================================
\section{Cadre général du pricing d'options}

\subsection{Portefeuille sans arbitrage et mesure risque--neutre}

On se place dans un marché frictionless composé :
\begin{itemize}
  \item d'un actif sans risque $B_t$ vérifiant $dB_t = r B_t\,dt$, donc
        $B_t = e^{rt}$ pour un taux sans risque constant $r$ ;
  \item d'un actif risqué $S_t$ (l'action sous--jacente).
\end{itemize}

Le principe d'absence d'arbitrage, couplé avec la complétude du marché
dans le modèle de Black--Scholes, garantit l'existence d'une \emph{mesure
risque--neutre} $\mathbb{Q}$ sous laquelle le prix escompté d'un actif
financier est une martingale. Pour un dérivé de payoff $f(S_T)$ à la
maturité $T$, son prix à la date $t$ est
\begin{equation}
  V(t,S_t)
  = \mathbb{E}^{\mathbb{Q}}\Bigl[
      e^{-r(T-t)} f(S_T)
      \,\big|\, \mathcal{F}_t
    \Bigr].
\end{equation}

En particulier, à la date initiale $t=0$ :
\begin{equation}
  V(0,S_0)
  = \mathbb{E}^{\mathbb{Q}}\bigl[
      e^{-rT} f(S_T)
    \bigr].
\end{equation}

Nous allons voir plusieurs façons de calculer (ou d'approximer)
cette quantité.

\subsection{Modèle Black--Scholes en temps continu}

Sous la mesure risque--neutre, on suppose que le sous--jacent suit
un mouvement brownien géométrique avec dividende continu $d$ :
\begin{equation}
  dS_t = (r-d) S_t\,dt + \sigma S_t\,dW_t,
\end{equation}
où $\sigma>0$ est la volatilité et $W_t$ un mouvement brownien standard.

La solution explicite de cette E.D.S. est donnée par
\begin{equation}
  S_t = S_0 \exp\Bigl(
    \bigl(r-d-\tfrac12\sigma^2\bigr)t + \sigma W_t
  \Bigr).
\end{equation}

Ce modèle, bien que simplificateur, permet d'obtenir :
\begin{itemize}
  \item des formules fermées (Black--Scholes) pour certaines options vanille ;
  \item une PDE par la méthode de réplication ;
  \item une distribution lognormale explicite de $S_T$, utile pour Monte Carlo.
\end{itemize}

Dans le projet, on rencontre aussi le modèle de Heston, où la variance
du sous--jacent est elle--même stochastique. Mathématiquement, les idées
de base restent similaires mais les calculs deviennent plus lourds ;
Monte Carlo est alors particulièrement naturel.

%======================================================================
\section{Pricing par PDE et schéma de Crank--Nicolson}

Cette section détaille la construction mathématique du pricer basé sur
la PDE de Black--Scholes et sa discrétisation par un schéma de
Crank--Nicolson en log--prix. C'est la base de l'onglet
\og Bermuda / Barrier (Crank--Nicolson) \fg{} de l'application.

\subsection{PDE de Black--Scholes}

On note $V(t,S)$ le prix d'une option en $t$ lorsque $S_t=S$.
En appliquant la formule d'Itô au portefeuille auto--finançant qui
rédplique le payoff, on obtient la PDE de Black--Scholes :
\begin{equation}
  \partial_t V
  + \tfrac12 \sigma^2 S^2 \partial_{SS} V
  + (r-d)S\partial_S V
  - r V = 0,
  \qquad 0\le t < T,\ S>0,
\end{equation}
avec condition terminale
\begin{equation}
  V(T,S) = f(S),
\end{equation}
où $f$ est le payoff (par exemple $(S-K)^+$ pour un call).

Cette PDE exprime l'équilibre entre la dynamique stochastique de $S_t$ et
la gestion d'un portefeuille de couverture.

\subsection{Changement de variables en log--prix}

Afin de rendre le problème plus stable numériquement, on introduit la
variable de log--prix
\begin{equation}
  x = \ln S,\qquad S=e^x,
\end{equation}
et on définit une nouvelle fonction
\begin{equation}
  u(t,x) = V(t,e^x).
\end{equation}

En utilisant la règle de dérivation en chaîne, on a :
\begin{align}
  \partial_S V(t,S)
  &= \frac{1}{S}\,\partial_x u(t,x),\\
  \partial_{SS} V(t,S)
  &= \frac{1}{S^2}\bigl(\partial_{xx} u(t,x) - \partial_x u(t,x)\bigr).
\end{align}

En remplaçant dans la PDE, on obtient, pour $u$,
\begin{equation}
  \partial_t u
  + \tfrac12 \sigma^2 \partial_{xx}u
  + \bigl(r-d - \tfrac12 \sigma^2\bigr)\partial_x u
  - r u = 0.
\end{equation}

Intuition :
\begin{itemize}
  \item passer en log--prix homogénéise les variations de $S$ (mêmes
        increments en $x$ correspondent à des variations relatives
        constantes en $S$) ;
  \item la diffusion en $x$ est \og plus régulière \fg{}, ce qui améliore
        la stabilité du schéma numérique.
\end{itemize}

\subsection{Temps restant à courir}

On pose le temps inversé
\begin{equation}
  \tau = T - t, \qquad U(\tau,x) = u(T-\tau,x).
\end{equation}
Alors $\partial_\tau U = -\partial_t u$ et la PDE devient
\begin{equation}
  \partial_\tau U
  = \tfrac12 \sigma^2 \partial_{xx}U
  + \mu_X \partial_x U
  - r U,
\end{equation}
où
\begin{equation}
  \mu_X = r-d - \tfrac12 \sigma^2.
\end{equation}

La condition terminale $u(T,x) = f(e^x)$ devient une condition
\emph{initiale}
\begin{equation}
  U(0,x) = f(e^x).
\end{equation}

Nous avons donc transformé le problème en un problème de diffusion
parabolique en temps croissant $\tau$.

\subsection{Discrétisation en espace et en temps}

On choisit un intervalle borné pour $x$ :
\[
  x \in [-x_{\max}, x_{\max}],
\]
avec typiquement $x_{\max} \approx 5\sigma\sqrt{T}$, ce qui permet de couvrir
la plupart des mouvements possibles de $S$. On introduit une grille
uniforme :
\begin{align}
  x_i &= -x_{\max} + i\Delta x,\qquad i=0,\dots,N,\\
  \tau_n &= n\Delta\tau,\qquad n=0,\dots,J,\quad \Delta\tau = T/J.
\end{align}
On note $U_i^n \approx U(\tau_n, x_i)$.

Les dérivées spatiales sont approximées par différences finies centrales :
\begin{align}
  \partial_x U(\tau_n,x_i)
  &\approx \frac{U_{i+1}^n - U_{i-1}^n}{2\Delta x},\\[0.3em]
  \partial_{xx} U(\tau_n,x_i)
  &\approx \frac{U_{i+1}^n - 2U_i^n + U_{i-1}^n}{\Delta x^2}.
\end{align}

\subsection{Schéma de Crank--Nicolson}

Le schéma de Crank--Nicolson est une méthode implicite
à \emph{ordre 2 en temps}, construite comme la moyenne de l'Euler implicite
et de l'Euler explicite appliqués à l'opérateur spatial.

On note $\mathcal{L}$ l'opérateur différentiel
\begin{equation}
  \mathcal{L}U
  = \tfrac12 \sigma^2 \partial_{xx}U
    + \mu_X \partial_x U
    - r U.
\end{equation}
Son approximation en différences finies sur la grille est notée
$\mathcal{L}_h$.

Le schéma de Crank--Nicolson s'écrit :
\begin{equation}
  \frac{U_i^{n+1} - U_i^n}{\Delta \tau}
  = \tfrac12 \bigl(\mathcal{L}_h U^{n+1}\bigr)_i
    + \tfrac12 \bigl(\mathcal{L}_h U^n\bigr)_i.
\end{equation}

En développant, on obtient, pour chaque $i$, une combinaison linéaire
de $U_{i-1}^{n+1}$, $U_i^{n+1}$, $U_{i+1}^{n+1}$ égale à une combinaison
connue de $U_{i-1}^{n}$, $U_i^{n}$, $U_{i+1}^{n}$. En notant
$(U^n)_i = U_i^n$, on obtient finalement un système matriciel
\begin{equation}
  A\,U^{n+1} = B\,U^{n},
\end{equation}
où $A$ et $B$ sont des matrices tridiagonales construites à partir de
coefficients $\{a_i,b_i,c_i\}$ (pour $A$) et de leurs opposés ou dérivés
pour $B$.

Dans le code, ces coefficients sont représentés par des vecteurs
$(a,b,c)$, puis assemblés en matrices denses $A$ et $B$. Le pas de temps
se fait par
\begin{equation}
  U^{n+1} = A^{-1} B U^n.
\end{equation}

\subsection{Payoff et conditions aux limites}

\paragraph{Payoff terminal.}
Pour un call européen de strike $K$ :
\begin{equation}
  f(S) = (S-K)^+ = \max(S-K,0),
\end{equation}
et pour un put :
\begin{equation}
  f(S) = (K-S)^+.
\end{equation}
Sur la grille en $x$, la condition initiale en temps $\tau=0$ est
\begin{equation}
  U_i^0 = f(e^{x_i}),\qquad i=0,\dots,N.
\end{equation}

\paragraph{Conditions aux bornes en $x$.}
On doit également imposer des conditions aux limites pour $x=x_0$ et
$x=x_N$ (les bords gauche et droit de la grille).

Pour un call :
\begin{itemize}
  \item lorsque $S\to 0$ ($x\to -\infty$), l'option vaut $\approx 0$,
        donc on impose $U(\tau,x_0)\approx 0$ ;
  \item lorsque $S\to +\infty$ ($x\to +\infty$), on a asymptotiquement
        $V \sim S$, donc une condition du type
        \[
          U(\tau,x_N) \approx e^{x_N} - K e^{-r\tau}
        \]
        est utilisée.
\end{itemize}

Pour un put, les rôles sont inversés :
\begin{itemize}
  \item pour $S\to 0$, $V \approx K e^{-r\tau}$ ;
  \item pour $S\to +\infty$, $V\to 0$.
\end{itemize}

Ces conditions sont appliquées explicitement dans le code avant ou après
chaque résolution de $A U^{n+1} = B U^n$.

\subsection{Options américaines et bermudéennes}

Pour une option américaine, l'exercice est autorisé en tout temps
continu $t\in[0,T]$, mais en pratique on ne peut le considérer que
sur une grille de dates.

\paragraph{Point de vue théorie des arrêts.}
Soit $\mathcal{T}_{t,T}$ l'ensemble des temps d'arrêt à valeurs dans
$[t,T]$. Le prix vérifie
\begin{equation}
  V(t,S)
  = \sup_{\tau\in\mathcal{T}_{t,T}}
    \mathbb{E}^{\mathbb{Q}}\bigl[
      e^{-r(\tau-t)} f(S_\tau)
      \,\big|\, S_t=S
    \bigr].
\end{equation}
Dans un cadre discret ($t=t_0 < t_1 < \dots < t_J=T$), l'exercice
est restreint à ces dates.

\paragraph{Implémentation numérique.}
Dans le schéma CN, après avoir calculé $U^n$ par
\[
  U^n = A^{-1} B U^{n-1},
\]
on applique à chaque noeud $i$ la projection
\begin{equation}
  U_i^n \leftarrow \max\bigl(U_i^n, f(e^{x_i})\bigr),
\end{equation}
qui impose la contrainte que la valeur de l'option ne peut être
inférieure au payoff immédiat (exercice instantané).

Pour une option \emph{bermudéenne}, l'exercice n'est permis qu'à un
sous--ensemble des dates de la grille :
\[
  0 < t_{n_1} < t_{n_2} < \dots < t_{n_k}=T.
\]
La projection $\max(\cdot,f)$ n'est appliquée que pour les indices
$n\in\{n_1,\dots,n_k\}$. Dans le code, c'est implémenté via un paramètre
du type \texttt{exercise\_step}.

\subsection{Options barrières (up--and--out, double knock--out)}

Les options barrières sont path--dépendantes : leur payoff dépend de
la trajectoire maximale ou minimale du sous--jacent. Par exemple :
\begin{itemize}
  \item \emph{up--and--out} : l'option est annulée si le sous--jacent
        franchit une barrière haute $H_u$ avant l'échéance ;
  \item \emph{double knock--out} : l'option est annulée si $S_t$
        sort d'un couloir $[H_d,H_u]$.
\end{itemize}

Dans le schéma CN en log--prix, on impose cette condition de manière
discrète :
\begin{itemize}
  \item pour l'up--and--out, on annule $U_i^n$ dès que $e^{x_i} \ge H_u$ ;
  \item pour le double knock--out, on annule $U_i^n$ si
        $e^{x_i} \le H_d$ ou $e^{x_i} \ge H_u$.
\end{itemize}

Cela vaut à la fois sur la condition terminale (payoff) et à chaque
pas de temps en remontant vers l'origine.

\subsection{Extraction du prix et des grecs}

Après avoir effectué les $J$ pas de temps, on obtient un vecteur $U^J$ qui
représente la solution à la date initiale. Le prix pour $S_0$ est
obtenu en prenant le noeud $x_{i^\star}$ tel que $e^{x_{i^\star}}\approx S_0$
(souvent l'indice central) :
\begin{equation}
  \text{Price} \approx U_{i^\star}^J.
\end{equation}

Les sensibilités (\emph{grecs}) sont approchées par différences finies
autour de ce point :
\begin{align}
  \Delta
  &\approx \frac{U_{i^\star+1}^J - U_{i^\star-1}^J}{
    S_0 e^{\Delta x} - S_0 e^{-\Delta x}},\\[0.3em]
  \Gamma
  &\approx \frac{
      \dfrac{U_{i^\star+1}^J - U_{i^\star}^J}{S_0 e^{\Delta x} - S_0}
      -
      \dfrac{U_{i^\star}^J - U_{i^\star-1}^J}{S_0 - S_0 e^{-\Delta x}}
    }{
      \tfrac12\bigl(S_0 e^{\Delta x} - S_0 e^{-\Delta x}\bigr)
    },\\[0.3em]
  \Theta
  &\approx -\frac{U_{i^\star}^J - U_{i^\star}^{J-1}}{\Delta t},
\end{align}
où $U^{J-1}$ est la solution à l'instant de temps immédiatement suivant.

\medskip
Cette partie correspond au contenu détaillé de \texttt{math\_bermuda\_barrier.tex}
et au module de pricing PDE utilisé dans l'onglet \og Bermuda / Barrier \fg{}
de l'application.

%======================================================================
\section{Pricing par Monte Carlo}

Nous abordons maintenant la méthode de Monte Carlo, très flexible,
utilisée pour estimer numériquement les espérances sous $\mathbb{Q}$.

\subsection{Monte Carlo classique pour une option européenne}

On considère une option de payoff $f(S_T)$. Le prix théorique est
\begin{equation}
  V_0 = \mathbb{E}^{\mathbb{Q}}\bigl[e^{-rT} f(S_T)\bigr].
\end{equation}

L'idée Monte Carlo est de générer $n$ trajectoires indépendantes
$(S_t^{(k)})_{0\le t\le T}$, puis de calculer l'estimateur
\begin{equation}
  \hat{V}_n
  = e^{-rT}\,\frac{1}{n} \sum_{k=1}^n f\bigl(S_T^{(k)}\bigr).
\end{equation}

Cet estimateur est sans biais et, sous des hypothèses standard, converge
vers $V_0$ lorsque $n\to\infty$ (loi des grands nombres).

\subsection{Erreur statistique et intervalle de confiance}

On définit
\begin{equation}
  X_k = e^{-rT} f\bigl(S_T^{(k)}\bigr),
\end{equation}
supposées i.i.d.\ avec variance $\sigma_X^2$. L'estimateur est
\begin{equation}
  \hat{V}_n = \frac{1}{n} \sum_{k=1}^n X_k.
\end{equation}

Le théorème central limite assure que
\begin{equation}
  \sqrt{n}\,\frac{\hat{V}_n - V_0}{\hat{\sigma}_X}
  \xrightarrow[n\to\infty]{\mathcal{L}} \mathcal{N}(0,1),
\end{equation}
où $\hat{\sigma}_X$ est l'écart--type empirique des $X_k$.

Par conséquent, un intervalle de confiance asymptotique à niveau $1-\alpha$
est
\begin{equation}
  \hat{V}_n \pm z_{1-\alpha/2}\,\frac{\hat{\sigma}_X}{\sqrt{n}},
\end{equation}
avec $z_{1-\alpha/2}$ le quantile de la loi normale standard.

Dans le projet, la fonction \texttt{monte\_carlo\_simulation} implémente
exactement cette idée, en affichant le prix estimé et l'intervalle de
confiance.

\subsection{Simulation de trajectoires}

Pour le GBM, on utilise la solution explicite
\begin{equation}
  S_{t_{j+1}}^{(k)}
  = S_{t_j}^{(k)} \exp\Bigl(
      (r - d - \tfrac12\sigma^2)\Delta t
      + \sigma \sqrt{\Delta t}\,Z_{j+1}^{(k)}
    \Bigr),
\end{equation}
avec $Z_{j+1}^{(k)}\sim\mathcal{N}(0,1)$ indépendants, et
$\Delta t = T/m$.

Pour Heston, les équations étant plus complexes, on utilise des schémas
numériques (par exemple de type Euler ou Milstein) pour simuler à la fois
$S_t$ et la variance $v_t$.

%======================================================================
\section{Algorithme de Longstaff--Schwartz}

La méthode de Longstaff--Schwartz (2001) est une technique Monte Carlo
pour valoriser des options américaines ou bermudéennes, c'est-à-dire
des options avec exercice anticipé.

\subsection{Problème d'optimisation d'arrêt optimal}

On considère une suite de dates d'exercice possibles
$0 = t_0 < t_1 < \dots < t_m = T$ et un payoff $f_j(S_{t_j})$ à la
date $t_j$. Par exemple, pour un put américain,
\begin{equation}
  f_j(S) = (K-S)^+.
\end{equation}

On cherche à maximiser
\begin{equation}
  V_0
  = \sup_{\tau\in\mathcal{T}}
    \mathbb{E}^{\mathbb{Q}}\Bigl[
      e^{-r\tau} f_\tau(S_\tau)
    \Bigr],
\end{equation}
où $\mathcal{T}$ est l'ensemble des temps d'arrêt à valeurs dans
$\{t_0,\dots,t_m\}$.

\subsection{Idée de régression sur la valeur de continuation}

La stratégie optimale se base sur la comparaison, à chaque date $t_j$,
entre :
\begin{itemize}
  \item le payoff d'exercice immédiat $f_j(S_{t_j})$ ;
  \item la valeur de continuation
    \[
      C_j(S_{t_j}) =
      \mathbb{E}^{\mathbb{Q}}\bigl[
        e^{-r\Delta t} V_{j+1}(S_{t_{j+1}})\,\big|\, S_{t_j}
      \bigr].
    \]
\end{itemize}

L'idée de Longstaff--Schwartz est d'approximer $C_j(\cdot)$ par une régression
sur une base de fonctions (polynômes) à partir de trajectoires simulées.

\subsection{Description pas à pas}

On simule $n$ trajectoires $S_{t_j}^{(k)}$, $k=1,\dots,n$, $j=0,\dots,m$.
\begin{enumerate}
  \item À maturité $t_m=T$, on pose
    \[
      V_m^{(k)} = f_m\bigl(S_{t_m}^{(k)}\bigr).
    \]
  \item Pour $j=m-1,\dots,1$ (remontée en temps) :
    \begin{enumerate}
      \item On identifie les trajectoires où l'option est dans la monnaie :
        \[
          \mathcal{I}_j = \bigl\{k : f_j(S_{t_j}^{(k)}) > 0\bigr\}.
        \]
      \item Pour ces trajectoires, on considère les cash--flows actualisés
        \[
          Y^{(k)} = e^{-r\Delta t} V_{j+1}^{(k)}.
        \]
      \item On approxime $C_j$ par une régression
        \[
          Y^{(k)} \approx \sum_{\ell=0}^p \beta_{j,\ell}\,
          \phi_\ell\bigl(S_{t_j}^{(k)}\bigr),\qquad k\in\mathcal{I}_j,
        \]
        où $\{\phi_\ell\}$ est une base (par exemple polynômes en $S$) et
        $(\beta_{j,\ell})$ sont déterminés par moindres carrés.
      \item Pour chaque trajectoire $k$, on calcule
        \[
          \hat{C}_j^{(k)} =
          \sum_{\ell=0}^p \beta_{j,\ell}\,
          \phi_\ell\bigl(S_{t_j}^{(k)}\bigr).
        \]
      \item Décision d'exercice :
        \[
          V_j^{(k)} =
          \begin{cases}
            f_j(S_{t_j}^{(k)}), &
              \text{si } f_j(S_{t_j}^{(k)}) \ge \hat{C}_j^{(k)},\\[0.3em]
            e^{-r\Delta t}\,V_{j+1}^{(k)}, &
              \text{sinon.}
          \end{cases}
        \]
    \end{enumerate}
  \item Enfin, le prix est la moyenne des cash--flows actualisés obtenus
    sur les trajectoires :
    \[
      \hat{V}_0 = \frac{1}{n}\sum_{k=1}^n V_0^{(k)}.
    \]
\end{enumerate}

Dans le code, on utilise la fonction \texttt{numpy.polynomial.Polynomial.fit}
pour ajuster un polynôme de degré 5 aux données $(S_{t_j}^{(k)},Y^{(k)})$,
ce qui donne directement une approximation de $C_j$.

%======================================================================
\section{Modèle binomial de Cox--Ross--Rubinstein (CRR)}

Le modèle CRR fournit une approximation discrète du mouvement brownien
géométrique via un arbre binomial.

\subsection{Construction de l'arbre}

On divise la période $[0,T]$ en $n$ intervalles de longueur $\Delta t = T/n$.
À chaque pas, le sous--jacent peut monter (up) ou descendre (down) de facteurs
\begin{equation}
  u = e^{\sigma\sqrt{\Delta t}},
  \qquad
  d = \frac{1}{u}.
\end{equation}

La valeur du sous--jacent au pas $j$ et dans l'état $i$ est
\begin{equation}
  S_{j,i} = S_0\,u^i d^{j-i},
  \qquad 0\le i\le j.
\end{equation}

Sous la mesure risque--neutre, la probabilité d'un mouvement \og up \fg{}
est choisie de manière à reproduire la dérive $r$ :
\begin{equation}
  p = \frac{e^{r\Delta t} - d}{u-d},\qquad q=1-p.
\end{equation}

\subsection{Valorisation par remontée dans l'arbre}

Pour une option européenne de payoff $f(S_T)$, on fixe
\begin{equation}
  V_n^i = f(S_{n,i}),\qquad i=0,\dots,n.
\end{equation}
Puis, pour $j=n-1,\dots,0$,
\begin{equation}
  V_j^i = e^{-r\Delta t}
  \bigl(p V_{j+1}^{i+1} + q V_{j+1}^i\bigr).
\end{equation}
Le prix initial est $V_0^0$.

Pour une option américaine, on ajoute l'exercice anticipé :
\begin{equation}
  V_j^i
  = \max\Bigl(
      f(S_{j,i}),
      e^{-r\Delta t}\bigl(p V_{j+1}^{i+1} + q V_{j+1}^i\bigr)
    \Bigr).
\end{equation}

Dans le projet, la fonction \texttt{crr\_pricing} implémente précisément
ce schéma pour un put américain.

%======================================================================
\section{Options barrières et lookback : formules fermées et PDE}

Nous regroupons ici les aspects spécifiques aux options barrières
et lookback, qui combinent techniques analytiques et EDP.

\subsection{Option barrière up--and--out}

On considère un call up--and--out avec barrière haute $B>K$. Le payoff est
\begin{equation}
  f(S_T) = (S_T-K)^+\,\mathbf{1}_{\{\max_{0\le u\le T} S_u < B\}}.
\end{equation}

Grâce au principe de réflexion pour le brownien, on peut exprimer le prix
comme combinaison de termes de type Black--Scholes. On introduit
les fonctions auxiliaires (dans le code : \texttt{delta\_p}, \texttt{delta\_m}) :
\begin{align}
  \delta_p(\tau, z, r, \sigma)
    &= \frac{\ln z + (r + \frac12\sigma^2)\tau}{\sigma\sqrt{\tau}},\\
  \delta_m(\tau, z, r, \sigma)
    &= \frac{\ln z + (r - \frac12\sigma^2)\tau}{\sigma\sqrt{\tau}}.
\end{align}

Une formule typique pour le call up--and--out est
\begin{align}
  V_{\text{UO}}(t,S_0)
  &= S_0\bigl[
      N(\delta_p(\tau, S_0/K))
      - N(\delta_p(\tau, S_0/B))
    \bigr]
  - K e^{-r\tau}
    \bigl[
      N(\delta_m(\tau, S_0/K))
      - N(\delta_m(\tau, S_0/B))
    \bigr]
  \nonumber\\
  &\quad
  - B\Bigl(\frac{S_0}{B}\Bigr)^{-\frac{2r}{\sigma^2}}
    \bigl[
      N(\delta_p(\tau, B^2/(K S_0)))
      - N(\delta_p(\tau, B/S_0))
    \bigr]
  \nonumber\\
  &\quad
  + K e^{-r\tau}
    \Bigl(\frac{S_0}{B}\Bigr)^{-\frac{2r}{\sigma^2}+1}
    \bigl[
      N(\delta_m(\tau, B^2/(K S_0)))
      - N(\delta_m(\tau, B/S_0))
    \bigr],
\end{align}
avec $\tau = T-t$. La méthode \texttt{price\_exact} implémente une
expression de ce type.

\subsection{EDP pour le call barrière}

La classe correspondante propose aussi une approche EDP, similaire à celle
du call européen, mais sur un domaine spatial tronqué $S\in[K/3,B]$ et
avec condition de Dirichlet $V=0$ sur la barrière $S=B$. Le schéma
utilise les mêmes idées de Crank--Nicolson avec résolution tridiagonale
par \texttt{const\_tri\_diag\_mat\_solve}.

\subsection{Option lookback à strike flottant}

Une option lookback à strike flottant a pour payoff
\begin{equation}
  f(S_T,M_T) = M_T - S_T,
\end{equation}
où $M_T = \max_{0\le u\le T} S_u$.

On travaille souvent avec le ratio
\begin{equation}
  Z_T = \frac{S_T}{M_T}\in(0,1],
\end{equation}
qui est invariant par homothétie de $S$. En posant $z=Z_T$, on peut écrire
le prix sous la forme
\begin{equation}
  V(t,S_0) = S_0\,v(z),
\end{equation}
et obtenir une EDP réduite pour $v$, ou une formule fermée en termes
de fonctions normales.

Dans le code, la méthode \texttt{price\_exact} utilise une expression
du type
\begin{align}
  v(z)
  &= (1+p) z\,N(\delta_p(\tau,z))
   + e^{-r\tau} N\bigl(-\delta_m(\tau,z)\bigr)
   - p e^{-r\tau} z^{1-p} N\bigl(-\delta_m(\tau,1/z)\bigr)
   - z,
\end{align}
avec $p = \sigma^2/(2r)$ et $\tau=T-t$, puis renvoie $V(t,S_0) = S_0 v(z)$.
Au temps initial, on a typiquement $z=1$ puisque $M_0=S_0$.

\subsection{EDP et schéma numérique pour la lookback}

Le pricer lookback dispose aussi d'une EDP en variables réduites
$(\tau,x)$, avec un opérateur proche de celui du call européen. Les
conditions aux frontières sont plus subtiles :
\begin{itemize}
  \item une condition Dirichlet sur une des frontières (valeur connue) ;
  \item une condition de Neumann (dérivée) sur l'autre bord ;
  \item un domaine spatial choisi en fonction de la variable de ratio.
\end{itemize}

Le système linéaire obtenu à chaque pas de temps est tridiagonal et
résolu via \texttt{tri\_diag\_mat\_solve\_arr}. Enfin, on reconvertit
le résultat dans les variables originales $(S,t)$ et on utilise une
interpolation linéaire (\texttt{get\_result}) pour obtenir le prix
exactement en $S_0$.

\subsection{Monte Carlo pour la lookback}

La méthode \texttt{price\_monte\_carlo} repose sur la simulation
directe des trajectoires de $S_t$ et de son maximum, ce qui nécessite
de simuler le brownien et son maximum conditionnel. Le payoff
$M_T - S_T$ est évalué trajectoire par trajectoire, puis actualisé et
moyenné.

%======================================================================
\section{Lien avec l'application Streamlit}

Le fichier \texttt{pricing\_app.py} rassemble les différents pricers
dans une interface interactive :
\begin{itemize}
  \item onglet \textbf{Bermuda / Barrier (Crank--Nicolson)} :
    utilisation du schéma PDE en log--prix pour des options européennes,
    américaines, bermudéennes et barrières, avec calcul des grecs.
  \item onglet \textbf{Longstaff (MC / LS / BSM / CRR)} :
    \begin{itemize}
      \item Monte Carlo classique,
      \item Monte Carlo Longstaff--Schwartz (américaine),
      \item formule de Black--Scholes--Merton (européenne),
      \item arbre CRR pour l'américaine.
    \end{itemize}
  \item onglet \textbf{Lookback / Barrier / European} :
    \begin{itemize}
      \item pricer analytique et EDP pour le call européen,
      \item pricer barrière up--and--out (formule fermée + PDE),
      \item pricer lookback à strike flottant (formule fermée + PDE + MC).
    \end{itemize}
\end{itemize}

Chaque ensemble de paramètres que l'utilisateur choisit dans l'interface
se traduit en paramètres d'entrée pour les fonctions et classes décrites
dans ces notes. On peut ainsi visualiser de manière interactive :
\begin{itemize}
  \item l'impact de la volatilité, du taux, du nombre de pas de temps,
        du nombre de trajectoires sur le prix et la précision ;
  \item les différences entre méthodes (PDE vs MC vs CRR) ;
  \item l'effet de l'exercice anticipé (américaine/bermudéenne) ou des
        barrières et de la mémoire du maximum (lookback).
\end{itemize}

\bigskip
Ces notes fournissent la \emph{charpente mathématique} sous--jacente
au code : elles peuvent servir de support de cours ou de base pour
faire évoluer les modèles (volatilité locale, sauts, etc.).

\end{document}

