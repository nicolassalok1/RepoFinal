\documentclass[11pt,a4paper]{article}
\usepackage[utf8]{inputenc}
\usepackage[T1]{fontenc}
\usepackage[french]{babel}
\usepackage{amsmath,amssymb,amsfonts}
\usepackage{bm}
\usepackage{geometry}
\geometry{margin=2.5cm}

\title{Mathématiques des pricers Lookback, Barrière et Européenne}
\author{}
\date{}

\begin{document}

\maketitle

\section{Modèle sous--jacent}

On suppose que le sous--jacent $S_t$ suit sous la mesure risque--neutre
une dynamique de type Black--Scholes avec taux sans risque $r$ et
volatilité constante $\sigma$ :
\begin{equation}
  dS_t = r S_t\,dt + \sigma S_t\,dW_t,
\end{equation}
où $W_t$ est un mouvement brownien standard.

On note
\begin{equation}
  M_t = \max_{0\le u\le t} S_u
\end{equation}
le maximum historique du sous--jacent jusqu'au temps $t$.

\section{Option européenne vanille}

La classe \texttt{european\_call\_option} implémente une option call
européenne classique de payoff
\begin{equation}
  f(S_T) = (S_T - K)^+.
\end{equation}

\subsection{Formule fermée de Black--Scholes}

Le prix à la date $t$ est
\begin{equation}
  C(t,S_t)
  = S_t N\bigl(d_1\bigr)
    - K e^{-r(T-t)} N\bigl(d_2\bigr),
\end{equation}
où
\begin{align}
  d_1 &= \frac{\ln(S_t/K) + (r + \frac12\sigma^2)(T-t)}{\sigma\sqrt{T-t}},\\
  d_2 &= d_1 - \sigma\sqrt{T-t},
\end{align}
et $N$ désigne la fonction de répartition de la loi normale standard.

La méthode \texttt{price\_exact} du code applique directement cette formule.

\subsection{PDE de Black--Scholes et schéma CN}

La même classe propose une résolution numérique par équation aux dérivées
partielles (EDP) avec un schéma de Crank--Nicolson sur une grille en log--prix
et en temps transformé. On introduit les variables
\begin{equation}
  x = \ln\frac{S}{K},\qquad
  \tau = \frac{\sigma^2}{2}(T-t),
\end{equation}
et la fonction réduite $u(\tau,x)$ telle que
\begin{equation}
  C(t,S) = K\,u\bigl(\tau, x\bigr).
\end{equation}
On montre alors que $u$ satisfait l'EDP
\begin{equation}
  \partial_\tau u = \partial_{xx} u + (1 + D)\,\partial_x u,
  \qquad D = \frac{2r}{\sigma^2},
\end{equation}
avec condition initiale (en $\tau=0$)
\begin{equation}
  u(0,x) = \max\bigl(0,1-e^{-x}\bigr),
\end{equation}
et des conditions aux frontières correspondantes aux comportements
asymptotiques du call.

Le code construit une grille uniforme en $\tau$ et en $x$, puis applique
un schéma en différences finies implicite de type Crank--Nicolson.
À chaque pas de temps, le système linéaire tridiagonal est résolu via
un algorithme spécialisé (\texttt{const\_tri\_diag\_mat\_solve}).

\section{Option barrière up--and--out}

La classe \texttt{barrier\_call\_option} traite un call up--and--out
de barrière $B > K$, dont le payoff est
\begin{equation}
  f(S_T) = (S_T-K)^+\,\mathbf{1}_{\{\max_{0\le u\le T} S_u < B\}}.
\end{equation}

\subsection{Formule fermée}

La distribution jointe de $(S_T, \max_{u\le T} S_u)$ sous Black--Scholes
se dérive grâce au principe de réflexion appliqué au mouvement brownien
en log--prix. On peut alors exprimer le prix de l'option barrière comme
une combinaison de fonctions de répartition normales.

On introduit les fonctions auxiliaires (voir \texttt{aux\_functions.py}) :
\begin{align}
  \delta_p(\tau, z, r, \sigma)
    &= \frac{\ln z + (r + \frac12\sigma^2)\tau}{\sigma\sqrt{\tau}},\\
  \delta_m(\tau, z, r, \sigma)
    &= \frac{\ln z + (r - \frac12\sigma^2)\tau}{\sigma\sqrt{\tau}}.
\end{align}

La fonction \texttt{price\_exact} du code implémente une formule analytique
du type
\begin{align}
  V_{\text{UO}}(t,S_0)
  &= S_0\bigl[
      N(\delta_p(\tau, S_0/K))
      - N(\delta_p(\tau, S_0/B))
    \bigr]
  - K e^{-r\tau}
    \bigl[
      N(\delta_m(\tau, S_0/K))
      - N(\delta_m(\tau, S_0/B))
    \bigr]
  \nonumber\\
  &\quad
  - B\Bigl(\frac{S_0}{B}\Bigr)^{-\frac{2r}{\sigma^2}}
    \bigl[
      N(\delta_p(\tau, B^2/(K S_0)))
      - N(\delta_p(\tau, B/S_0))
    \bigr]
  \nonumber\\
  &\quad
  + K e^{-r\tau}
    \Bigl(\frac{S_0}{B}\Bigr)^{-\frac{2r}{\sigma^2}+1}
    \bigl[
      N(\delta_m(\tau, B^2/(K S_0)))
      - N(\delta_m(\tau, B/S_0))
    \bigr],
\end{align}
avec $\tau = T-t$. Cette expression résulte d'une combinaison de la
formule de Black--Scholes standard et de termes \og image \fg{} associés
à la probabilité de franchissement de la barrière.

\subsection{EDP et schéma numérique}

La même classe propose une approximation par EDP pour le prix du call
barrière, en travaillant dans la variable réduite
$x = \ln(S/K)$ sur l'intervalle $S \in [K/3,B]$.
La fonction $u(\tau,x)$ satisfait à la même EDP que la call européenne
vanille :
\begin{equation}
  \partial_\tau u = \partial_{xx} u + (1 + D)\,\partial_x u,
  \qquad D = \frac{2r}{\sigma^2},
\end{equation}
mais avec la condition de Dirichlet $u(\tau,x_B)=0$ correspondant à
$S=B$ (barrière).

Le schéma de Crank--Nicolson est appliqué sur la grille en $x$, et l'équation
linéaire tridiagonale résultante est résolue à chaque pas de temps via
la routine \texttt{const\_tri\_diag\_mat\_solve}, comme pour le call
européen.

\section{Option lookback à strike flottant}

La classe \texttt{lookback\_call\_option} traite un call de type lookback
à strike flottant, dont le payoff à maturité est
\begin{equation}
  f(S_T, M_T) = M_T - S_T.
\end{equation}
Cette option permet de \og regarder en arrière \fg{} sur tout l'historique
de $S_t$ pour fixer le strike au meilleur prix observé.

\subsection{Formule fermée}

On introduit la variable réduite
\begin{equation}
  Z_T = \frac{S_T}{M_T} \in (0,1],
\end{equation}
et on travaille avec $z = Z_T$ comme variable d'état.
Sous Black--Scholes, la distribution de $Z_T$ (ou, plus précisément, la
distribution jointe de $(S_T,M_T)$) peut être obtenue via le principe de
réflexion pour un brownien arithmétique dérivé de $\ln S_t$.

La méthode \texttt{price\_exact} utilise une expression analytique de la
forme
\begin{equation}
  V(t,S_0)
  = S_0\,v(z),\qquad z = \frac{S_T}{M_T},
\end{equation}
où, pour $\tau = T-t$,
\begin{align}
  p &= \frac{\sigma^2}{2r},\\
  v(z)
  &= (1+p) z\,N\bigl(\delta_p(\tau,z)\bigr)
   + e^{-r\tau} N\bigl(-\delta_m(\tau,z)\bigr)
   - p e^{-r\tau} z^{1-p} N\bigl(-\delta_m(\tau,1/z)\bigr)
   - z.
\end{align}
Au temps initial $t=0$, on a typiquement $z=1$ (puisque $M_0=S_0$),
ce qui donne le prix initial.

Cette formule est cohérente avec les expressions classiques de la littérature
pour les lookbacks à strike flottant.

\subsection{EDP pour la lookback}

La classe propose également une résolution par EDP dans des variables
réduites. On introduit des variables $(\tau,x)$ adaptées à la symétrie
de l'équation (voir \texttt{lookback\_call.py}), conduisant à une EDP
de la forme
\begin{equation}
  \partial_\tau u = \partial_{xx}u + (1 + D)\,\partial_x u,
  \qquad D = \frac{2r}{\sigma^2},
\end{equation}
sur un domaine approprié en $x$.

Les conditions aux frontières sont plus délicates que pour un call
vanille et reflètent le fait qu'on travaille sur le ratio $S/M$.
Numériquement, le code :
\begin{itemize}
  \item impose une condition de type Dirichlet sur un bord,
  \item impose une condition de Neumann (dérivée) sur l'autre bord,
  \item résout à chaque pas de temps un système tridiagonal par
    \texttt{tri\_diag\_mat\_solve\_arr}.
\end{itemize}
Après la résolution, on revient aux variables originales $(S,t)$.
Le prix au temps initial s'obtient en interpolant la solution de la grille
au point $z=1$ et en remultipliant par $S_0$ (méthode \texttt{get\_pde\_result}).

\subsection{Méthode de Monte Carlo}

La méthode \texttt{price\_monte\_carlo} simule directement des trajectoires
de $S_t$ et de son maximum, via la représentation
\begin{equation}
  S_t = S_0 \exp\Bigl( (r - \tfrac12\sigma^2)t + \sigma W_t\Bigr).
\end{equation}
On génère un brownien conditionnellement à une valeur terminale donnée
et on utilise la loi du maximum conditionnel d'un brownien.
Le payoff $M_T - S_T$ est alors évalué sur chaque trajectoire, puis
actualisé (par $e^{-rT}$) et moyenné.

\section{Interpolation sur la grille}

Pour les schémas EDP, le prix au point $(S_0,t_0)$ est rarement situé
exactement sur un noeud de la grille spatiale. La fonction utilitaire
\texttt{get\_result} effectue une interpolation linéaire :
si $x_{\ell} < S_0 < x_r$ sont les deux noeuds encadrant $S_0$, et
$(v_\ell,v_r)$ les valeurs de la solution en ces points, alors
\begin{equation}
  V(S_0,t)
  \approx v_\ell + (v_r - v_\ell)
    \frac{S_0 - x_\ell}{x_r - x_\ell}.
\end{equation}

Cette étape est utilisée par toutes les méthodes \texttt{get\_pde\_result}
pour renvoyer un prix continu à partir d'une solution sur grille.

\end{document}

