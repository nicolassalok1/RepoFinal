\documentclass[11pt,a4paper]{article}
\usepackage[utf8]{inputenc}
\usepackage[T1]{fontenc}
\usepackage[french]{babel}
\usepackage{amsmath,amssymb,amsfonts}
\usepackage{bm}
\usepackage{geometry}
\geometry{margin=2.5cm}

\title{Mathématiques des pricers Monte Carlo, Longstaff--Schwartz, Black--Scholes et CRR}
\author{}
\date{}

\begin{document}

\maketitle

\section{Cadre général de valorisation}

On considère un actif risqué $S_t$ sous la mesure risque--neutre $\mathbb{Q}$,
défini sur $(\Omega,\mathcal{F},\mathbb{Q})$, et un taux sans risque constant $r$.
Le prix d'une option européenne de payoff $f(S_T)$ à la maturité $T$ est donné par
\begin{equation}
  V_0 = \mathbb{E}^{\mathbb{Q}}\bigl[e^{-rT} f(S_T)\bigr].
\end{equation}

Dans la bibliothèque \texttt{Longstaff}, le sous--jacent peut suivre :
\begin{itemize}
  \item un mouvement brownien géométrique (GBM) : 
    \[
      dS_t = \mu S_t\,dt + \sigma S_t\,dW_t,
    \]
  \item un modèle de Heston :
    \[
      \begin{cases}
        dS_t = \mu S_t\,dt + \sqrt{v_t}\,S_t\,dW_t^S,\\[0.2em]
        dv_t = \kappa(\theta - v_t)\,dt + \eta \sqrt{v_t}\,dW_t^v,
      \end{cases}
    \]
    avec $\mathrm{corr}(dW_t^S,dW_t^v) = \rho$.
\end{itemize}
Dans un cadre rigoureux, la dérive $\mu$ doit coïncider avec $r$ sous $\mathbb{Q}$,
mais l'implémentation laisse la liberté de paramétrer $\mu$.

\section{Monte Carlo classique}

\subsection{Estimateur de prix}

On simule $n$ trajectoires indépendantes du processus $(S_t)_{0\le t\le T}$,
discrétisées sur une grille $0 = t_0 < t_1 < \dots < t_m = T$.
Pour une option de payoff $f(S_T)$, l'estimateur Monte Carlo du prix est
\begin{equation}
  \hat{V}_n
  = e^{-rT}\,\frac{1}{n}\sum_{k=1}^n f\bigl(S_T^{(k)}\bigr).
\end{equation}

Dans le code (\texttt{monte\_carlo\_simulation}), on note
la dérive par \texttt{process.mu}; l'actualisation est effectuée via
$\exp(-\mu T)$, ce qui repose sur l'approximation $\mu \approx r$.

\subsection{Erreur statistique et intervalle de confiance}

Soient les variables aléatoires
\begin{equation}
  X_k = e^{-rT} f\bigl(S_T^{(k)}\bigr),\qquad k=1,\dots,n,
\end{equation}
supposées i.i.d. de variance finie $\sigma_X^2$.
On a
\begin{equation}
  \hat{V}_n = \frac{1}{n}\sum_{k=1}^n X_k
  \xrightarrow[n\to\infty]{a.s.} V_0.
\end{equation}

Le théorème central limite donne
\begin{equation}
  \sqrt{n}\,\frac{\hat{V}_n - V_0}{\hat{\sigma}_X}
  \xrightarrow[n\to\infty]{\mathcal{L}} \mathcal{N}(0,1),
\end{equation}
où $\hat{\sigma}_X$ est l'estimateur empirique de l'écart--type.
Un intervalle de confiance asymptotique à niveau $1-\alpha$ est donc
\begin{equation}
  \hat{V}_n \pm z_{1-\alpha/2}\,\frac{\hat{\sigma}_X}{\sqrt{n}},
\end{equation}
où $z_{1-\alpha/2}$ est le quantile de la loi normale standard. C'est
exactement la formule utilisée dans le code.

\section{Algorithme de Longstaff--Schwartz}

Pour valoriser une option américaine (ou plus généralement une option
à exercice bermudéen) avec payoff discret $f_j(S_{t_j})$ aux dates
$0=t_0<t_1<\dots<t_m=T$, on cherche la valeur
\begin{equation}
  V_0
  = \sup_{\tau\in \mathcal{T}}\,
    \mathbb{E}^{\mathbb{Q}}\Bigl[
      e^{-r\tau} f_\tau(S_\tau)
    \Bigr],
\end{equation}
où $\mathcal{T}$ est l'ensemble des temps d'arrêt à valeurs dans
$\{t_0,\dots,t_m\}$.

L'idée de Longstaff--Schwartz (2001) est d'approximer, par régression,
les valeurs de continuation conditionnelles
\begin{equation}
  C_j(s)
  = \mathbb{E}^{\mathbb{Q}}\Bigl[
      e^{-r\Delta t} V_{j+1}(S_{t_{j+1}})
      \,\big|\, S_{t_j}=s
    \Bigr],
\end{equation}
où $\Delta t = t_{j+1}-t_j$.

\subsection{Algorithme}

On note $S_{t_j}^{(k)}$ la valeur du sous--jacent sur la $k$-ième
trajectoire au temps $t_j$ ($k=1,\dots,n$). L'algorithme s'écrit :
\begin{enumerate}
  \item Simuler $n$ trajectoires complètes $S_{t_0}^{(k)},\dots,S_{t_m}^{(k)}$.
  \item Initialiser à maturité
    \[
      V_m^{(k)} = f_m\bigl(S_{t_m}^{(k)}\bigr),\qquad k=1,\dots,n.
    \]
  \item Pour $j=m-1,\dots,1$ (remontée en temps) :
    \begin{enumerate}
      \item Identifier les trajectoires \emph{dans la monnaie} à $t_j$ :
        \[
          \mathcal{I}_j
          = \bigl\{k:\ f_j(S_{t_j}^{(k)}) > 0\bigr\}.
        \]
      \item Pour ces trajectoires, calculer les cash--flows actualisés
        \[
          Y^{(k)} = e^{-r\Delta t}\,V_{j+1}^{(k)}.
        \]
      \item Approximer $C_j(s)$ par une régression linéaire
        sur une base de fonctions $\{\phi_\ell\}_{\ell=0}^p$ (polynômes) :
        \[
          Y^{(k)} \approx \sum_{\ell=0}^p \beta_{j,\ell}\,\phi_\ell\bigl(S_{t_j}^{(k)}\bigr),
          \qquad k\in\mathcal{I}_j.
        \]
        Le vecteur de coefficients $\bm{\beta}_j$ est obtenu par moindres
        carrés ordinaires.
      \item Pour chaque trajectoire $k$, on évalue la continuation estimée
        \[
          \hat{C}_j^{(k)}
          = \sum_{\ell=0}^p \beta_{j,\ell}\,\phi_\ell\bigl(S_{t_j}^{(k)}\bigr).
        \]
      \item Décision d'exercice :
        \[
          V_j^{(k)}
          =
          \begin{cases}
            f_j\bigl(S_{t_j}^{(k)}\bigr),
              &\text{si } f_j(S_{t_j}^{(k)}) \ge \hat{C}_j^{(k)},\\[0.3em]
            e^{-r\Delta t}\,V_{j+1}^{(k)},
              &\text{sinon.}
          \end{cases}
        \]
    \end{enumerate}
  \item Le prix de l'option est enfin donné par
    \[
      \hat{V}_0
      = \frac{1}{n}\sum_{k=1}^n e^{-r t_{j^\star(k)}}\,
        f_{j^\star(k)}\bigl(S_{t_{j^\star(k)}}^{(k)}\bigr),
    \]
    où $t_{j^\star(k)}$ est la date d'exercice optimale sur la trajectoire $k$.
\end{enumerate}

Dans le code, l'approximation de $C_j$ utilise \texttt{numpy.polynomial.Polynomial.fit}
sur des puissances de $S_{t_j}$ jusqu'à un degré $5$.

\section{Formule de Black--Scholes--Merton}

Sous le modèle de Black--Scholes, avec $S_t$ suivant
\begin{equation}
  dS_t = r S_t\,dt + \sigma S_t\,dW_t,
\end{equation}
le prix d'un call européen de strike $K$ et de maturité $T$ est
\begin{equation}
  C_{\mathrm{BS}}(S_0,K,T)
  = S_0 N(d_1) - K e^{-rT}N(d_2),
\end{equation}
où
\begin{align}
  d_1 &= \frac{\ln(S_0/K) + (r + \frac12\sigma^2)T}{\sigma\sqrt{T}},\\
  d_2 &= d_1 - \sigma\sqrt{T},
\end{align}
et $N$ est la fonction de répartition de la loi normale standard.

Le prix du put correspondant se déduit de la parité put--call :
\begin{equation}
  P_{\mathrm{BS}}(S_0,K,T)
  = C_{\mathrm{BS}}(S_0,K,T) - S_0 + K e^{-rT}.
\end{equation}

La fonction \texttt{black\_scholes\_merton} du code implémente directement
ces formules, en retournant le prix du call ou du put selon le paramètre
booléen \texttt{option.call}.

\section{Modèle binomial de Cox--Ross--Rubinstein (CRR)}

Le modèle CRR approxime le mouvement brownien géométrique par un arbre
binomial à $n$ pas sur l'intervalle $[0,T]$. On pose
\begin{align}
  \Delta t &= \frac{T}{n},\\
  u &= e^{\sigma \sqrt{\Delta t}},\\
  d &= \frac{1}{u}.
\end{align}
La valeur du sous--jacent au pas $j$ et dans l'état $i$ est
\begin{equation}
  S_{j,i} = S_0\,u^i d^{j-i},\qquad 0\le i\le j.
\end{equation}

Sous la mesure risque--neutre, la probabilité d'un mouvement \og up \fg{}
est
\begin{equation}
  p = \frac{e^{r\Delta t} - d}{u-d},
\end{equation}
et $q = 1-p$ est la probabilité d'un mouvement \og down \fg{}.

\subsection{Valorisation d'une option américaine}

On initialise les payoffs à maturité, par exemple pour un put américain
\begin{equation}
  V_n^i = \bigl(K - S_{n,i}\bigr)^+,\qquad i=0,\dots,n.
\end{equation}
Ensuite, on remonte dans l'arbre par récurrence :
\begin{equation}
  V_j^i
  = \max\Bigl(
      \bigl(K - S_{j,i}\bigr)^+,
      e^{-r\Delta t}\,\bigl(p V_{j+1}^{i+1} + q V_{j+1}^i\bigr)
    \Bigr),
  \qquad j = n-1,\dots,0.
\end{equation}
La première composante représente le payoff d'exercice immédiat, la seconde
la continuation espérée sous $\mathbb{Q}$.

Le prix de l'option à la date $0$ est alors
\begin{equation}
  V_0^0.
\end{equation}

La fonction \texttt{crr\_pricing} du code suit exactement ce schéma :
construction des $S_{n,i}$ à maturité, calcul des payoffs, puis remontée
avec la règle de mise à jour ci--dessus.

\end{document}

